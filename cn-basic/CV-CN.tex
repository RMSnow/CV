%----------------------------------------------------------------------------------------
%	PACKAGES AND OTHER DOCUMENT CONFIGURATIONS
%----------------------------------------------------------------------------------------

\documentclass[letterpaper,AutoFakeBold]{twentysecondcv} % a4paper for A4
\usepackage{CJKutf8}
\usepackage{booktabs}

% 设置双倍行距
\linespread{1.6}

%----------------------------------------------------------------------------------------
%	 PERSONAL INFORMATION
%----------------------------------------------------------------------------------------

% If you don't need one or more of the below, just remove the content leaving the command, e.g. \cvnumberphone{}

\profilepic{1.jpg} % Profile picture

\cvname{张雪遥} % Your name
\cvjobtitle{武汉大学计算机学院} % Job title/career

\cvdate{软件工程专业\,2015\,级本科生} % Date of birth
\cvaddress{} % Short address/location, use \newline if more than 1 line is required
\cvnumberphone{+86-15203900168} % Phone number
\cvsite{http://www.zhangxueyao.com} % Personal website
\cvmail{xueyao\_98@foxmail.com} % Email address

%----------------------------------------------------------------------------------------

\begin{document}

% 楷体
% \begin{CJK*}{UTF8}{gkai}
% 宋体
\begin{CJK*}{UTF8}{gbsn}

%----------------------------------------------------------------------------------------
%	 ABOUT ME
%----------------------------------------------------------------------------------------

% \aboutme{Alice is a sensible prepubescent girl from a wealthy English family who finds herself in a strange world ruled by imagination and fantasy. Alice feels comfortable with her identity and has a strong sense that her environment is comprised of clear, logical, and consistent rules and features. Alice's familiarity with the world has led one critic to describe her as a "disembodied intellect". Alice displays great curiosity and attempts to fit her diverse experiences into a clear understanding of the world.} % To have no About Me section, just remove all the text and leave \aboutme{}
\aboutme{
	张雪遥,男,1998\,年\,1\,月\,17\,日生.
	\\2015\,年\,9\,月进入武汉大学,就读于计算机学院软件工程专业,曾获
	\href{https://raw.githubusercontent.com/RMSnow/CV/master/materials/NationaScholarship.jpg}
	{国家奖学金\,(2016)\,},并多次获得武汉大学优秀学生甲等奖学金、武汉大学三好学生等荣誉.
	\\[2ex]现于\href{http://nisplab.whu.edu.cn/index.html}{武汉大学网络信息系统安全实验室(\emph{NIS\&P})}实习,
	指导教师为\href{http://iqua.ece.toronto.edu/ychen/}{陈艳姣}教授,研究课题包含用户点击率预测、广告欺骗检测等.
	\\[2ex]研究兴趣包括数据挖掘、机器学习等,希望能够在研究生期间,投身人工智能相关领域.
}

%----------------------------------------------------------------------------------------

\makeprofile % Print the sidebar

\begin{CJK*}{UTF8}{gkai}
\section{\large本科教育}
\end{CJK*}

\begin{center}
	\begin{tabular}{lll}
		\toprule
		\textbf{GPA} & \quad3.83 & \quad满绩:4.0 \\
		\midrule
		\textbf{平均分} & \quad90.373 & \quad满分:100 \\
		\midrule
		\textbf{排名} & \quad1\% & \quad专业人数:246 \\
		\midrule
		\textbf{CET-6} & \quad522 & \quad过线分:425 \\
		\bottomrule
	\end{tabular}
\end{center}

% \begin{twenty}
% 	\twentyitem{GPA}{3.83}{}{}
% 	\twentyitem{Grades}{90.373}{}{}
% \end{twenty}

%----------------------------------------------------------------------------------------

\begin{CJK*}{UTF8}{gkai}
\section{\large获奖经历}
\end{CJK*}

% TODO:加粗
\begin{itemize}
	\setlength{\itemsep}{0pt}
	\setlength{\parsep}{0pt}
	\setlength{\parskip}{0pt}
	\item 2016\,-\,2017\,学年度武汉大学优秀学生甲等奖学金(注:校级一等,前5\%).
	\item 2016\,-\,2017\,学年度武汉大学三好学生(注:校级,前1\%).
	\item 武汉大学\,2016\,年度优秀团员(注:校级,前1\%).
	\item \href{https://raw.githubusercontent.com/RMSnow/CV/master/materials/NationaScholarship.jpg}
	{2015\,-\,2016\,学年度国家奖学金(注:国家级,前1\%)}.
	\item 2015\,-\,2016\,学年度武汉大学优秀学生甲等奖学金(注:校级一等,前5\%).
	\item 2015\,-\,2016\,学年度武汉大学三好学生(注:校级,前1\%).
	\item 2015\,-\,2015\,学年度武汉大学优秀学生干部(注:校级,前1\%).
	\item 2014\,年全国高中生数学联赛一等奖.
\end{itemize}

%----------------------------------------------------------------------------------------

\begin{CJK*}{UTF8}{gkai}
\section{\large研究经历}
\end{CJK*}

2018.03\,起 \qquad \href{http://nisplab.whu.edu.cn/index.html}{武汉大学网络信息系统安全实验室(\emph{NIS\&P})}\hfill本科生实习
\begin{itemize}
	% \setlength{\itemsep}{0pt}
	% \setlength{\parsep}{0pt}
	% \setlength{\parskip}{0pt}
	\item 指导教师:\href{http://iqua.ece.toronto.edu/ychen/}{陈艳姣}教授.
	\item 研究课题:用户点击率预测\,\emph{(Click-Through Rate, CTR)}、
	广告欺骗检测\,\emph{(Ad Fraud Detection)}、
	推荐系统\,\emph{(Recommendation System)}\,等.
	\item 研究方法:数据挖掘、机器学习、深度学习等.
\end{itemize}

%----------------------------------------------------------------------------------------

\begin{CJK*}{UTF8}{gkai}
\section{\large课程经历}
\end{CJK*}

\begin{enumerate}
	\setlength{\itemsep}{0pt}
	\setlength{\parsep}{0pt}
	\setlength{\parskip}{0pt}
	\item 2018\,年春季:人机交互,课程助教
	\item 2017\,年秋季:\href{http://www.javatree.cn/}{面向对象程序设计\,(\emph{Java})\,},课程助教
	\item \emph{Coursera\,}在线课程:\href{https://www.coursera.org/specializations/deep-learning}{\emph{Deep Learning Specialization}}
		\begin{itemize}
			\item 该专项培训共分为:
			\emph{Neural Networks and Deep Learning},
			\emph{Improving Deep Neural Networks},
			\emph{Structuring Machine Learning Projects},
			\emph{Convolutional Neural Networks},
			\emph{Sequence Models}\,等五个课程.
			\item 该专项培训由\,\emph{deeplearning.ai}\,发布,主讲人为\,\href{https://www.coursera.org/instructor/andrewng}{\emph{Andrew Ng}}\,.
			\item 成功通过课程考试,并拿到\,\emph{Coursera}\,证书.
		\end{itemize}
\end{enumerate}

%----------------------------------------------------------------------------------------

\begin{CJK*}{UTF8}{gkai}
	\section{\large社会活动}
	\end{CJK*}
	
	\begin{itemize}
		\setlength{\itemsep}{0pt}
		\setlength{\parsep}{0pt}
		\setlength{\parskip}{0pt}
		\item 2016.05\,-\,2017.05 \qquad 院学生会文艺部部长.
		\item 2016.03\,-\,2016.10 \qquad 武汉大学自强工作室\href{https://taoke.ziqiang.net.cn/}{淘课啦}项目组,产品副经理.
		\item 2015.09\,-\, \qquad \qquad \qquad 软件工程五班班长.
	\end{itemize}

%----------------------------------------------------------------------------------------
\newpage % Start a new page
\makeprofile % Print the sidebar
%----------------------------------------------------------------------------------------

\begin{CJK*}{UTF8}{gkai}
\section{\large项目与比赛}
\end{CJK*}

2018.03-2018.05 \quad \emph{TalkingData AdTracking Fraud Detection Challenge} \hfill Top 30\%
\begin{itemize}
	\setlength{\itemsep}{0pt}
	\setlength{\parsep}{0pt}
	\setlength{\parskip}{0pt}
	\item 该比赛发布在\,\href{https://www.kaggle.com/c/talkingdata-adtracking-fraud-detection#description}{\emph{Kaggle}\,}平台,
	由\,\href{https://www.talkingdata.com/}{\emph{TalkingData}}\,提供数据集.
	\item 比赛的最终目标,是通过用户行为的数据集构建算法,预测一个点击广告的用户是否会下载\,APP\,.
	\item 我们尝试了机器学习、深度学习的多种模型,如\,\emph{LR}\,、\,\emph{FM}\,、
	\,\emph{GBDT}\,、\,\emph{DNN}\,等,最终构建了基于\,\emph{GBDT/LR}\,,
	以及\,\emph{DNN/LR/FM}\,等混合模型.
	\item 比赛尚未结束,目前参赛队伍约为\,3500\,支,我们的当前排名为\,30\%.
\end{itemize}

2018.04-2018.06 \qquad  蓝桥杯编程大赛 \hfill 湖北省一等奖
\begin{itemize}
	\setlength{\itemsep}{0pt}
	\setlength{\parsep}{0pt}
	\setlength{\parskip}{0pt}
	\item \href{http://dasai.lanqiao.cn/}{蓝桥杯大赛}的主办方为工业和信息化部人才交流中心.
	\item 参赛者需在4个小时的时间内解决若干算法问题.
	\item 比赛尚未结束,目前我已获得湖北省一等奖,并成功入围全国总决赛.
\end{itemize}

2017.10-2018.03 \qquad “云课堂”——在线教育平台开发 \hfill \emph{Java} 后端开发者
\begin{itemize}
	\setlength{\itemsep}{0pt}
	\setlength{\parsep}{0pt}
	\setlength{\parskip}{0pt}
	\item \href{https://www.dxtwangxiao.com/#/overview}{云课堂}是一个跨平台的在线教育应用,项目的开发团队约为\,30\,人.
	\item 整个应用采用\href{https://en.wikipedia.org/wiki/Microservices}{微服务架构},我们采用了\,\emph{Java}\,作为后端开发语言.
	\item 我作为后端开发者独立完成了“日志服务”模块,其中包含日志记录、流量监测与分析等功能.
\end{itemize}

%----------------------------------------------------------------------------------------

\vspace{250pt}
\profilesection{}

\begin{CJK*}{UTF8}{gkai}
\section{\large附录}
\end{CJK*}

\begin{itemize}
	\setlength{\itemsep}{0pt}
	\setlength{\parsep}{0pt}
	\setlength{\parskip}{0pt}
	\item 中英文成绩单:\href{https://raw.githubusercontent.com/RMSnow/CV/master/materials/AcademicTranscript.jpg}
	{武汉大学学生成绩单}
	\item 绩点证明:\href{https://raw.githubusercontent.com/RMSnow/CV/master/materials/GPA.jpg}
	{武汉大学全日制普通本科生学分绩点换算方法}
	\item \href{https://github.com/RMSnow/CV/blob/master/materials/XueyaoZhang_Awards.pdf}
	{获奖证明}
\end{itemize}

%----------------------------------------------------------------------------------------
%	 SECOND PAGE EXAMPLE
%----------------------------------------------------------------------------------------

% \newpage % Start a new page

% \makeprofile % Print the sidebar

% \section{Other information}

%\subsection{Review}

%Alice approaches Wonderland as an anthropologist, but maintains a strong sense of noblesse oblige that comes with her class status. She has confidence in her social position, education, and the Victorian virtue of good manners. Alice has a feeling of entitlement, particularly when comparing herself to Mabel, whom she declares has a ``poky little house," and no toys. Additionally, she flaunts her limited information base with anyone who will listen and becomes increasingly obsessed with the importance of good manners as she deals with the rude creatures of Wonderland. Alice maintains a superior attitude and behaves with solicitous indulgence toward those she believes are less privileged.

%\section{Other information}

%\subsection{Review}

%Alice approaches Wonderland as an anthropologist, but maintains a strong sense of noblesse oblige that comes with her class status. She has confidence in her social position, education, and the Victorian virtue of good manners. Alice has a feeling of entitlement, particularly when comparing herself to Mabel, whom she declares has a ``poky little house," and no toys. Additionally, she flaunts her limited information base with anyone who will listen and becomes increasingly obsessed with the importance of good manners as she deals with the rude creatures of Wonderland. Alice maintains a superior attitude and behaves with solicitous indulgence toward those she believes are less privileged.

%----------------------------------------------------------------------------------------
\end{CJK*}
\end{document} 
